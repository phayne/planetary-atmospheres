
% Default to the notebook output style

    


% Inherit from the specified cell style.




    
\documentclass[11pt]{article}

    
    
    \usepackage[T1]{fontenc}
    % Nicer default font (+ math font) than Computer Modern for most use cases
    \usepackage{mathpazo}

    % Basic figure setup, for now with no caption control since it's done
    % automatically by Pandoc (which extracts ![](path) syntax from Markdown).
    \usepackage{graphicx}
    % We will generate all images so they have a width \maxwidth. This means
    % that they will get their normal width if they fit onto the page, but
    % are scaled down if they would overflow the margins.
    \makeatletter
    \def\maxwidth{\ifdim\Gin@nat@width>\linewidth\linewidth
    \else\Gin@nat@width\fi}
    \makeatother
    \let\Oldincludegraphics\includegraphics
    % Set max figure width to be 80% of text width, for now hardcoded.
    \renewcommand{\includegraphics}[1]{\Oldincludegraphics[width=.8\maxwidth]{#1}}
    % Ensure that by default, figures have no caption (until we provide a
    % proper Figure object with a Caption API and a way to capture that
    % in the conversion process - todo).
    \usepackage{caption}
    \DeclareCaptionLabelFormat{nolabel}{}
    \captionsetup{labelformat=nolabel}

    \usepackage{adjustbox} % Used to constrain images to a maximum size 
    \usepackage{xcolor} % Allow colors to be defined
    \usepackage{enumerate} % Needed for markdown enumerations to work
    \usepackage{geometry} % Used to adjust the document margins
    \usepackage{amsmath} % Equations
    \usepackage{amssymb} % Equations
    \usepackage{textcomp} % defines textquotesingle
    % Hack from http://tex.stackexchange.com/a/47451/13684:
    \AtBeginDocument{%
        \def\PYZsq{\textquotesingle}% Upright quotes in Pygmentized code
    }
    \usepackage{upquote} % Upright quotes for verbatim code
    \usepackage{eurosym} % defines \euro
    \usepackage[mathletters]{ucs} % Extended unicode (utf-8) support
    \usepackage[utf8x]{inputenc} % Allow utf-8 characters in the tex document
    \usepackage{fancyvrb} % verbatim replacement that allows latex
    \usepackage{grffile} % extends the file name processing of package graphics 
                         % to support a larger range 
    % The hyperref package gives us a pdf with properly built
    % internal navigation ('pdf bookmarks' for the table of contents,
    % internal cross-reference links, web links for URLs, etc.)
    \usepackage{hyperref}
    \usepackage{longtable} % longtable support required by pandoc >1.10
    \usepackage{booktabs}  % table support for pandoc > 1.12.2
    \usepackage[inline]{enumitem} % IRkernel/repr support (it uses the enumerate* environment)
    \usepackage[normalem]{ulem} % ulem is needed to support strikethroughs (\sout)
                                % normalem makes italics be italics, not underlines
    

    
    
    % Colors for the hyperref package
    \definecolor{urlcolor}{rgb}{0,.145,.698}
    \definecolor{linkcolor}{rgb}{.71,0.21,0.01}
    \definecolor{citecolor}{rgb}{.12,.54,.11}

    % ANSI colors
    \definecolor{ansi-black}{HTML}{3E424D}
    \definecolor{ansi-black-intense}{HTML}{282C36}
    \definecolor{ansi-red}{HTML}{E75C58}
    \definecolor{ansi-red-intense}{HTML}{B22B31}
    \definecolor{ansi-green}{HTML}{00A250}
    \definecolor{ansi-green-intense}{HTML}{007427}
    \definecolor{ansi-yellow}{HTML}{DDB62B}
    \definecolor{ansi-yellow-intense}{HTML}{B27D12}
    \definecolor{ansi-blue}{HTML}{208FFB}
    \definecolor{ansi-blue-intense}{HTML}{0065CA}
    \definecolor{ansi-magenta}{HTML}{D160C4}
    \definecolor{ansi-magenta-intense}{HTML}{A03196}
    \definecolor{ansi-cyan}{HTML}{60C6C8}
    \definecolor{ansi-cyan-intense}{HTML}{258F8F}
    \definecolor{ansi-white}{HTML}{C5C1B4}
    \definecolor{ansi-white-intense}{HTML}{A1A6B2}

    % commands and environments needed by pandoc snippets
    % extracted from the output of `pandoc -s`
    \providecommand{\tightlist}{%
      \setlength{\itemsep}{0pt}\setlength{\parskip}{0pt}}
    \DefineVerbatimEnvironment{Highlighting}{Verbatim}{commandchars=\\\{\}}
    % Add ',fontsize=\small' for more characters per line
    \newenvironment{Shaded}{}{}
    \newcommand{\KeywordTok}[1]{\textcolor[rgb]{0.00,0.44,0.13}{\textbf{{#1}}}}
    \newcommand{\DataTypeTok}[1]{\textcolor[rgb]{0.56,0.13,0.00}{{#1}}}
    \newcommand{\DecValTok}[1]{\textcolor[rgb]{0.25,0.63,0.44}{{#1}}}
    \newcommand{\BaseNTok}[1]{\textcolor[rgb]{0.25,0.63,0.44}{{#1}}}
    \newcommand{\FloatTok}[1]{\textcolor[rgb]{0.25,0.63,0.44}{{#1}}}
    \newcommand{\CharTok}[1]{\textcolor[rgb]{0.25,0.44,0.63}{{#1}}}
    \newcommand{\StringTok}[1]{\textcolor[rgb]{0.25,0.44,0.63}{{#1}}}
    \newcommand{\CommentTok}[1]{\textcolor[rgb]{0.38,0.63,0.69}{\textit{{#1}}}}
    \newcommand{\OtherTok}[1]{\textcolor[rgb]{0.00,0.44,0.13}{{#1}}}
    \newcommand{\AlertTok}[1]{\textcolor[rgb]{1.00,0.00,0.00}{\textbf{{#1}}}}
    \newcommand{\FunctionTok}[1]{\textcolor[rgb]{0.02,0.16,0.49}{{#1}}}
    \newcommand{\RegionMarkerTok}[1]{{#1}}
    \newcommand{\ErrorTok}[1]{\textcolor[rgb]{1.00,0.00,0.00}{\textbf{{#1}}}}
    \newcommand{\NormalTok}[1]{{#1}}
    
    % Additional commands for more recent versions of Pandoc
    \newcommand{\ConstantTok}[1]{\textcolor[rgb]{0.53,0.00,0.00}{{#1}}}
    \newcommand{\SpecialCharTok}[1]{\textcolor[rgb]{0.25,0.44,0.63}{{#1}}}
    \newcommand{\VerbatimStringTok}[1]{\textcolor[rgb]{0.25,0.44,0.63}{{#1}}}
    \newcommand{\SpecialStringTok}[1]{\textcolor[rgb]{0.73,0.40,0.53}{{#1}}}
    \newcommand{\ImportTok}[1]{{#1}}
    \newcommand{\DocumentationTok}[1]{\textcolor[rgb]{0.73,0.13,0.13}{\textit{{#1}}}}
    \newcommand{\AnnotationTok}[1]{\textcolor[rgb]{0.38,0.63,0.69}{\textbf{\textit{{#1}}}}}
    \newcommand{\CommentVarTok}[1]{\textcolor[rgb]{0.38,0.63,0.69}{\textbf{\textit{{#1}}}}}
    \newcommand{\VariableTok}[1]{\textcolor[rgb]{0.10,0.09,0.49}{{#1}}}
    \newcommand{\ControlFlowTok}[1]{\textcolor[rgb]{0.00,0.44,0.13}{\textbf{{#1}}}}
    \newcommand{\OperatorTok}[1]{\textcolor[rgb]{0.40,0.40,0.40}{{#1}}}
    \newcommand{\BuiltInTok}[1]{{#1}}
    \newcommand{\ExtensionTok}[1]{{#1}}
    \newcommand{\PreprocessorTok}[1]{\textcolor[rgb]{0.74,0.48,0.00}{{#1}}}
    \newcommand{\AttributeTok}[1]{\textcolor[rgb]{0.49,0.56,0.16}{{#1}}}
    \newcommand{\InformationTok}[1]{\textcolor[rgb]{0.38,0.63,0.69}{\textbf{\textit{{#1}}}}}
    \newcommand{\WarningTok}[1]{\textcolor[rgb]{0.38,0.63,0.69}{\textbf{\textit{{#1}}}}}
    
    
    % Define a nice break command that doesn't care if a line doesn't already
    % exist.
    \def\br{\hspace*{\fill} \\* }
    % Math Jax compatability definitions
    \def\gt{>}
    \def\lt{<}
    % Document parameters
    \title{Lecture08\_Dynamics}
    
    
    

    % Pygments definitions
    
\makeatletter
\def\PY@reset{\let\PY@it=\relax \let\PY@bf=\relax%
    \let\PY@ul=\relax \let\PY@tc=\relax%
    \let\PY@bc=\relax \let\PY@ff=\relax}
\def\PY@tok#1{\csname PY@tok@#1\endcsname}
\def\PY@toks#1+{\ifx\relax#1\empty\else%
    \PY@tok{#1}\expandafter\PY@toks\fi}
\def\PY@do#1{\PY@bc{\PY@tc{\PY@ul{%
    \PY@it{\PY@bf{\PY@ff{#1}}}}}}}
\def\PY#1#2{\PY@reset\PY@toks#1+\relax+\PY@do{#2}}

\expandafter\def\csname PY@tok@w\endcsname{\def\PY@tc##1{\textcolor[rgb]{0.73,0.73,0.73}{##1}}}
\expandafter\def\csname PY@tok@c\endcsname{\let\PY@it=\textit\def\PY@tc##1{\textcolor[rgb]{0.25,0.50,0.50}{##1}}}
\expandafter\def\csname PY@tok@cp\endcsname{\def\PY@tc##1{\textcolor[rgb]{0.74,0.48,0.00}{##1}}}
\expandafter\def\csname PY@tok@k\endcsname{\let\PY@bf=\textbf\def\PY@tc##1{\textcolor[rgb]{0.00,0.50,0.00}{##1}}}
\expandafter\def\csname PY@tok@kp\endcsname{\def\PY@tc##1{\textcolor[rgb]{0.00,0.50,0.00}{##1}}}
\expandafter\def\csname PY@tok@kt\endcsname{\def\PY@tc##1{\textcolor[rgb]{0.69,0.00,0.25}{##1}}}
\expandafter\def\csname PY@tok@o\endcsname{\def\PY@tc##1{\textcolor[rgb]{0.40,0.40,0.40}{##1}}}
\expandafter\def\csname PY@tok@ow\endcsname{\let\PY@bf=\textbf\def\PY@tc##1{\textcolor[rgb]{0.67,0.13,1.00}{##1}}}
\expandafter\def\csname PY@tok@nb\endcsname{\def\PY@tc##1{\textcolor[rgb]{0.00,0.50,0.00}{##1}}}
\expandafter\def\csname PY@tok@nf\endcsname{\def\PY@tc##1{\textcolor[rgb]{0.00,0.00,1.00}{##1}}}
\expandafter\def\csname PY@tok@nc\endcsname{\let\PY@bf=\textbf\def\PY@tc##1{\textcolor[rgb]{0.00,0.00,1.00}{##1}}}
\expandafter\def\csname PY@tok@nn\endcsname{\let\PY@bf=\textbf\def\PY@tc##1{\textcolor[rgb]{0.00,0.00,1.00}{##1}}}
\expandafter\def\csname PY@tok@ne\endcsname{\let\PY@bf=\textbf\def\PY@tc##1{\textcolor[rgb]{0.82,0.25,0.23}{##1}}}
\expandafter\def\csname PY@tok@nv\endcsname{\def\PY@tc##1{\textcolor[rgb]{0.10,0.09,0.49}{##1}}}
\expandafter\def\csname PY@tok@no\endcsname{\def\PY@tc##1{\textcolor[rgb]{0.53,0.00,0.00}{##1}}}
\expandafter\def\csname PY@tok@nl\endcsname{\def\PY@tc##1{\textcolor[rgb]{0.63,0.63,0.00}{##1}}}
\expandafter\def\csname PY@tok@ni\endcsname{\let\PY@bf=\textbf\def\PY@tc##1{\textcolor[rgb]{0.60,0.60,0.60}{##1}}}
\expandafter\def\csname PY@tok@na\endcsname{\def\PY@tc##1{\textcolor[rgb]{0.49,0.56,0.16}{##1}}}
\expandafter\def\csname PY@tok@nt\endcsname{\let\PY@bf=\textbf\def\PY@tc##1{\textcolor[rgb]{0.00,0.50,0.00}{##1}}}
\expandafter\def\csname PY@tok@nd\endcsname{\def\PY@tc##1{\textcolor[rgb]{0.67,0.13,1.00}{##1}}}
\expandafter\def\csname PY@tok@s\endcsname{\def\PY@tc##1{\textcolor[rgb]{0.73,0.13,0.13}{##1}}}
\expandafter\def\csname PY@tok@sd\endcsname{\let\PY@it=\textit\def\PY@tc##1{\textcolor[rgb]{0.73,0.13,0.13}{##1}}}
\expandafter\def\csname PY@tok@si\endcsname{\let\PY@bf=\textbf\def\PY@tc##1{\textcolor[rgb]{0.73,0.40,0.53}{##1}}}
\expandafter\def\csname PY@tok@se\endcsname{\let\PY@bf=\textbf\def\PY@tc##1{\textcolor[rgb]{0.73,0.40,0.13}{##1}}}
\expandafter\def\csname PY@tok@sr\endcsname{\def\PY@tc##1{\textcolor[rgb]{0.73,0.40,0.53}{##1}}}
\expandafter\def\csname PY@tok@ss\endcsname{\def\PY@tc##1{\textcolor[rgb]{0.10,0.09,0.49}{##1}}}
\expandafter\def\csname PY@tok@sx\endcsname{\def\PY@tc##1{\textcolor[rgb]{0.00,0.50,0.00}{##1}}}
\expandafter\def\csname PY@tok@m\endcsname{\def\PY@tc##1{\textcolor[rgb]{0.40,0.40,0.40}{##1}}}
\expandafter\def\csname PY@tok@gh\endcsname{\let\PY@bf=\textbf\def\PY@tc##1{\textcolor[rgb]{0.00,0.00,0.50}{##1}}}
\expandafter\def\csname PY@tok@gu\endcsname{\let\PY@bf=\textbf\def\PY@tc##1{\textcolor[rgb]{0.50,0.00,0.50}{##1}}}
\expandafter\def\csname PY@tok@gd\endcsname{\def\PY@tc##1{\textcolor[rgb]{0.63,0.00,0.00}{##1}}}
\expandafter\def\csname PY@tok@gi\endcsname{\def\PY@tc##1{\textcolor[rgb]{0.00,0.63,0.00}{##1}}}
\expandafter\def\csname PY@tok@gr\endcsname{\def\PY@tc##1{\textcolor[rgb]{1.00,0.00,0.00}{##1}}}
\expandafter\def\csname PY@tok@ge\endcsname{\let\PY@it=\textit}
\expandafter\def\csname PY@tok@gs\endcsname{\let\PY@bf=\textbf}
\expandafter\def\csname PY@tok@gp\endcsname{\let\PY@bf=\textbf\def\PY@tc##1{\textcolor[rgb]{0.00,0.00,0.50}{##1}}}
\expandafter\def\csname PY@tok@go\endcsname{\def\PY@tc##1{\textcolor[rgb]{0.53,0.53,0.53}{##1}}}
\expandafter\def\csname PY@tok@gt\endcsname{\def\PY@tc##1{\textcolor[rgb]{0.00,0.27,0.87}{##1}}}
\expandafter\def\csname PY@tok@err\endcsname{\def\PY@bc##1{\setlength{\fboxsep}{0pt}\fcolorbox[rgb]{1.00,0.00,0.00}{1,1,1}{\strut ##1}}}
\expandafter\def\csname PY@tok@kc\endcsname{\let\PY@bf=\textbf\def\PY@tc##1{\textcolor[rgb]{0.00,0.50,0.00}{##1}}}
\expandafter\def\csname PY@tok@kd\endcsname{\let\PY@bf=\textbf\def\PY@tc##1{\textcolor[rgb]{0.00,0.50,0.00}{##1}}}
\expandafter\def\csname PY@tok@kn\endcsname{\let\PY@bf=\textbf\def\PY@tc##1{\textcolor[rgb]{0.00,0.50,0.00}{##1}}}
\expandafter\def\csname PY@tok@kr\endcsname{\let\PY@bf=\textbf\def\PY@tc##1{\textcolor[rgb]{0.00,0.50,0.00}{##1}}}
\expandafter\def\csname PY@tok@bp\endcsname{\def\PY@tc##1{\textcolor[rgb]{0.00,0.50,0.00}{##1}}}
\expandafter\def\csname PY@tok@fm\endcsname{\def\PY@tc##1{\textcolor[rgb]{0.00,0.00,1.00}{##1}}}
\expandafter\def\csname PY@tok@vc\endcsname{\def\PY@tc##1{\textcolor[rgb]{0.10,0.09,0.49}{##1}}}
\expandafter\def\csname PY@tok@vg\endcsname{\def\PY@tc##1{\textcolor[rgb]{0.10,0.09,0.49}{##1}}}
\expandafter\def\csname PY@tok@vi\endcsname{\def\PY@tc##1{\textcolor[rgb]{0.10,0.09,0.49}{##1}}}
\expandafter\def\csname PY@tok@vm\endcsname{\def\PY@tc##1{\textcolor[rgb]{0.10,0.09,0.49}{##1}}}
\expandafter\def\csname PY@tok@sa\endcsname{\def\PY@tc##1{\textcolor[rgb]{0.73,0.13,0.13}{##1}}}
\expandafter\def\csname PY@tok@sb\endcsname{\def\PY@tc##1{\textcolor[rgb]{0.73,0.13,0.13}{##1}}}
\expandafter\def\csname PY@tok@sc\endcsname{\def\PY@tc##1{\textcolor[rgb]{0.73,0.13,0.13}{##1}}}
\expandafter\def\csname PY@tok@dl\endcsname{\def\PY@tc##1{\textcolor[rgb]{0.73,0.13,0.13}{##1}}}
\expandafter\def\csname PY@tok@s2\endcsname{\def\PY@tc##1{\textcolor[rgb]{0.73,0.13,0.13}{##1}}}
\expandafter\def\csname PY@tok@sh\endcsname{\def\PY@tc##1{\textcolor[rgb]{0.73,0.13,0.13}{##1}}}
\expandafter\def\csname PY@tok@s1\endcsname{\def\PY@tc##1{\textcolor[rgb]{0.73,0.13,0.13}{##1}}}
\expandafter\def\csname PY@tok@mb\endcsname{\def\PY@tc##1{\textcolor[rgb]{0.40,0.40,0.40}{##1}}}
\expandafter\def\csname PY@tok@mf\endcsname{\def\PY@tc##1{\textcolor[rgb]{0.40,0.40,0.40}{##1}}}
\expandafter\def\csname PY@tok@mh\endcsname{\def\PY@tc##1{\textcolor[rgb]{0.40,0.40,0.40}{##1}}}
\expandafter\def\csname PY@tok@mi\endcsname{\def\PY@tc##1{\textcolor[rgb]{0.40,0.40,0.40}{##1}}}
\expandafter\def\csname PY@tok@il\endcsname{\def\PY@tc##1{\textcolor[rgb]{0.40,0.40,0.40}{##1}}}
\expandafter\def\csname PY@tok@mo\endcsname{\def\PY@tc##1{\textcolor[rgb]{0.40,0.40,0.40}{##1}}}
\expandafter\def\csname PY@tok@ch\endcsname{\let\PY@it=\textit\def\PY@tc##1{\textcolor[rgb]{0.25,0.50,0.50}{##1}}}
\expandafter\def\csname PY@tok@cm\endcsname{\let\PY@it=\textit\def\PY@tc##1{\textcolor[rgb]{0.25,0.50,0.50}{##1}}}
\expandafter\def\csname PY@tok@cpf\endcsname{\let\PY@it=\textit\def\PY@tc##1{\textcolor[rgb]{0.25,0.50,0.50}{##1}}}
\expandafter\def\csname PY@tok@c1\endcsname{\let\PY@it=\textit\def\PY@tc##1{\textcolor[rgb]{0.25,0.50,0.50}{##1}}}
\expandafter\def\csname PY@tok@cs\endcsname{\let\PY@it=\textit\def\PY@tc##1{\textcolor[rgb]{0.25,0.50,0.50}{##1}}}

\def\PYZbs{\char`\\}
\def\PYZus{\char`\_}
\def\PYZob{\char`\{}
\def\PYZcb{\char`\}}
\def\PYZca{\char`\^}
\def\PYZam{\char`\&}
\def\PYZlt{\char`\<}
\def\PYZgt{\char`\>}
\def\PYZsh{\char`\#}
\def\PYZpc{\char`\%}
\def\PYZdl{\char`\$}
\def\PYZhy{\char`\-}
\def\PYZsq{\char`\'}
\def\PYZdq{\char`\"}
\def\PYZti{\char`\~}
% for compatibility with earlier versions
\def\PYZat{@}
\def\PYZlb{[}
\def\PYZrb{]}
\makeatother


    % Exact colors from NB
    \definecolor{incolor}{rgb}{0.0, 0.0, 0.5}
    \definecolor{outcolor}{rgb}{0.545, 0.0, 0.0}



    
    % Prevent overflowing lines due to hard-to-break entities
    \sloppy 
    % Setup hyperref package
    \hypersetup{
      breaklinks=true,  % so long urls are correctly broken across lines
      colorlinks=true,
      urlcolor=urlcolor,
      linkcolor=linkcolor,
      citecolor=citecolor,
      }
    % Slightly bigger margins than the latex defaults
    
    \geometry{verbose,tmargin=1in,bmargin=1in,lmargin=1in,rmargin=1in}
    
    

    \begin{document}
    
    
    \maketitle
    
    

    
    \section{8. Dynamics of Planetary Atmospheres: Part
I}\label{dynamics-of-planetary-atmospheres-part-i}

\emph{Paul O. Hayne}

Motions in planetary atmospheres are governed fundamentally by force
balance. Different forces are dominant in different situations, so
usually the first step in tackling a problem in atmospheric dynamics is
to identify the most and least important of these forces for a given
system and timescale. Once we identify the appropriate dynamical regime,
we can solve the equations of motion (either analytically or
numerically, i.e. with a computer) to describe the resultant flow. Fluid
dynamics can be a complex and intimidating subject. Fortunately, many
phenomena in planetary atmospheres can be accurately described and
understood using a simplified set of equations appropriate to a
particular regime.

    \begin{Verbatim}[commandchars=\\\{\}]
{\color{incolor}In [{\color{incolor}11}]:} \PY{c+c1}{\PYZsh{} Toggle code on/off}
         \PY{k+kn}{from} \PY{n+nn}{IPython}\PY{n+nn}{.}\PY{n+nn}{display} \PY{k}{import} \PY{n}{HTML}
         \PY{c+c1}{\PYZsh{} This script provides a button to toggle display code on/off}
         \PY{n}{HTML}\PY{p}{(}\PY{l+s+s1}{\PYZsq{}\PYZsq{}\PYZsq{}}\PY{l+s+s1}{\PYZlt{}script\PYZgt{} code\PYZus{}show=true; }
         \PY{l+s+s1}{function code\PYZus{}toggle() }\PY{l+s+s1}{\PYZob{}}
         \PY{l+s+s1}{ if (code\PYZus{}show)}\PY{l+s+s1}{\PYZob{}}
         \PY{l+s+s1}{ \PYZdl{}(}\PY{l+s+s1}{\PYZsq{}}\PY{l+s+s1}{div.input}\PY{l+s+s1}{\PYZsq{}}\PY{l+s+s1}{).hide();}
         \PY{l+s+s1}{ \PYZcb{} else }\PY{l+s+s1}{\PYZob{}}
         \PY{l+s+s1}{ \PYZdl{}(}\PY{l+s+s1}{\PYZsq{}}\PY{l+s+s1}{div.input}\PY{l+s+s1}{\PYZsq{}}\PY{l+s+s1}{).show();}
         \PY{l+s+s1}{ \PYZcb{}}
         \PY{l+s+s1}{ code\PYZus{}show = !code\PYZus{}show}
         \PY{l+s+s1}{\PYZcb{} }
         \PY{l+s+s1}{\PYZdl{}( document ).ready(code\PYZus{}toggle);}
         \PY{l+s+s1}{    \PYZlt{}/script\PYZgt{}}
         \PY{l+s+s1}{    \PYZlt{}form action=}\PY{l+s+s1}{\PYZdq{}}\PY{l+s+s1}{javascript:code\PYZus{}toggle()}\PY{l+s+s1}{\PYZdq{}}\PY{l+s+s1}{\PYZgt{}\PYZlt{}input type=}\PY{l+s+s1}{\PYZdq{}}\PY{l+s+s1}{submit}\PY{l+s+s1}{\PYZdq{}}\PY{l+s+s1}{ value=}\PY{l+s+s1}{\PYZdq{}}\PY{l+s+s1}{Toggle code on/off}\PY{l+s+s1}{\PYZdq{}}\PY{l+s+s1}{\PYZgt{}\PYZlt{}/form\PYZgt{}}\PY{l+s+s1}{\PYZsq{}\PYZsq{}\PYZsq{}}\PY{p}{)}
\end{Verbatim}


\begin{Verbatim}[commandchars=\\\{\}]
{\color{outcolor}Out[{\color{outcolor}11}]:} <IPython.core.display.HTML object>
\end{Verbatim}
            
    \begin{Verbatim}[commandchars=\\\{\}]
{\color{incolor}In [{\color{incolor}2}]:} \PY{k+kn}{from} \PY{n+nn}{IPython}\PY{n+nn}{.}\PY{n+nn}{display} \PY{k}{import} \PY{n}{Image}\PY{p}{,} \PY{n}{HTML}
        \PY{n}{Image}\PY{p}{(}\PY{n}{filename}\PY{o}{=}\PY{l+s+s1}{\PYZsq{}}\PY{l+s+s1}{/Users/paha3326/main/Teaching/ASTR3720/hayne/lectures/animations/jupiter/jupiter\PYZhy{}voyager.gif.png}\PY{l+s+s1}{\PYZsq{}}\PY{p}{,} \PY{n}{width}\PY{o}{=}\PY{l+m+mi}{250}\PY{p}{)}
\end{Verbatim}

\texttt{\color{outcolor}Out[{\color{outcolor}2}]:}
    
    \begin{center}
    \adjustimage{max size={0.9\linewidth}{0.9\paperheight}}{output_2_0.png}
    \end{center}
    { \hspace*{\fill} \\}
    

    \textbf{Figure 8.1:} One month of images of Jupiter taken by NASA's
Voyager-1 spacecraft in January, 1979. The field of view is roughly
fixed on the "Great Red Spot" in the southern mid-latitudes, which
appears stationary despite the \$\sim\$10-hour rotation period. This
effect was created by using images acquired every 10 hr, with Jupiter at
the same local time in each frame.

\begin{center}\rule{0.5\linewidth}{\linethickness}\end{center}

    \subsection{8.1 Fundamentals of rotating
fluids}\label{fundamentals-of-rotating-fluids}

All planets rotate, as do their atmospheres. Fluid motions within the
atmosphere can be strongly affected by planetary rotation. So, we need
some tools to handle non-inertial forces. First, we describe the forces
affecting motion of a fluid in a rotating reference frame. After
identifying each force term, we look at ways to determine their relative
importance in order to solve problems.

\paragraph{Conventions:}\label{conventions}

Positions on the planet are given by three coordinates: \(X, Y, Z\),
with the planet's center of mass as the origin. For a spherical planet
with radius \(R\), the position vector is
\(\vec{R} = (X,Y,Z) = R(\cos\phi\cos\lambda, \cos\phi\sin\lambda, \sin\phi)\),
where \(\phi\) and \(\lambda\) are the latitude and longitude,
respectively. Unit vectors at each location \(\vec{R}\) define a local
coordinate system, pointing east-to-west (\(\hat{i}\)), south-to-north
(\(\hat{j}\)), and down-to-up (\(\hat{k}\)). Positions in this local
(rotating) reference frame are specified with respect to the local
coordinate system:
\(\vec{x} = (x,y,z) = x\hat{i} + y\hat{j} + z\hat{k}\). Wind velocity is
also a vector quantity,
\(\vec{v} = d\vec{x}/dt = (u,v,w) = u\hat{i} + v\hat{j} + w\hat{k}\).

    \begin{Verbatim}[commandchars=\\\{\}]
{\color{incolor}In [{\color{incolor}3}]:} \PY{n}{Image}\PY{p}{(}\PY{n}{filename}\PY{o}{=}\PY{l+s+s1}{\PYZsq{}}\PY{l+s+s1}{/Users/paha3326/main/Teaching/ASTR3720/hayne/lectures/lecture18/cartesian\PYZhy{}coords.png}\PY{l+s+s1}{\PYZsq{}}\PY{p}{,} \PY{n}{width}\PY{o}{=}\PY{l+m+mi}{250}\PY{p}{)}
\end{Verbatim}

\texttt{\color{outcolor}Out[{\color{outcolor}3}]:}
    
    \begin{center}
    \adjustimage{max size={0.9\linewidth}{0.9\paperheight}}{output_5_0.png}
    \end{center}
    { \hspace*{\fill} \\}
    

    \textbf{Figure 8.2:} Local cartesian coordinate system in a rotating
reference frame on a planet with radius \(R\) and rotation rate
\(\Omega\).

\begin{center}\rule{0.5\linewidth}{\linethickness}\end{center}

    \subsubsection{8.1.1 Force balance in the rotating
frame}\label{force-balance-in-the-rotating-frame}

Standing on the surface of the planet or floating in a balloon in the
atmosphere, we study atmospheric motions from within a rotating
reference frame. In this case, the equations of motion include both the
direct forces and the \emph{inertial} (or 'fictitious') forces due to
our rotating perspective. Newton's second law, \(m\vec{a} = \vec{F}\)
can be written

\begin{align}
    m\vec{a} ~~~~~&= ~~~~~~~~~~~~~(\mathrm{Direct~forces}) ~~~~~~~~~~~~~~~~~~~~ + ~~~~~~~~(\mathrm{Inertial~forces}) \\
    &= ~~~~~~~~~~~~~~~~~~~\vec{F}_\mathrm{direct} ~~~~~~~~~~~~~~~~~~~~~~~~~~~~ + ~~~~~~~ \vec{F}_\mathrm{inertial} ~~~~ \\
    &= ~~\vec{F}_\mathrm{gravity} + \vec{F}_\mathrm{pressure} + \vec{F}_\mathrm{friction}~~~~~~~~~~~+ ~~~ ~~~~\vec{F}_\mathrm{coriolis} + \vec{F}_\mathrm{centrifugal}
\end{align}

The first composite term (\(\vec{F}_\mathrm{direct}\)) on the right-hand
side (RHS) contains all of the forces acting on the mass within the
rotating system, including gravity, pressure, and friction. The second
composite term on the RHS is the sum of two inertial forces: (1)
\(\vec{F}_\mathrm{coriolis}\) is the Coriolis force, which arises due to
conservation of angular momentum; (2) \(\vec{F}_\mathrm{centrifugal}\)
is the centrifugal force, which acts in a direction perpendicular to the
axis of rotation. Below, we untangle each of these forces, then develop
approaches to estimating their relative magnitudes. We typically deal
with accelerations, \(\vec{a} = \vec{F}/m\), which of course are forces
per unit mass.

    \subsubsection{8.1.2 Pressure gradient
acceleration}\label{pressure-gradient-acceleration}

We previously encountered the concept of \emph{hydrostatic balance},
where gravity is balanced against the vertical pressure gradient.
Horizontal pressure gradients also produce accelerations, directed from
high to low pressure. For two air parcels with pressures \(p_1\) and
\(p_2\), separated by a small distance \(dx\), the \emph{pressure
gradient} is \(dp = p_2 - p_1\) and the acceleration is

\begin{align}
    a_x = -\frac{1}{\rho}\frac{dp}{dx}
\end{align}

where \(\rho\) is the local density (kg m\(^{-3}\)) of the fluid. The
minus sign is necessary, because the acceleration is directed "down" the
gradient, i.e. from high to low pressure (Fig. 8.3). Generalizing to
three dimensions, the pressure gradient imparts an acceleration

\begin{align}
    \vec{a} &= -\frac{1}{\rho}\vec{\nabla}p \\
    \mathrm{or} \\
    (a_x, a_y, a_z) &= -\frac{1}{\rho}\left(\frac{\partial p}{\partial x}, \frac{\partial p}{\partial y}, \frac{\partial p}{\partial z} \right)
\end{align}

    \begin{Verbatim}[commandchars=\\\{\}]
{\color{incolor}In [{\color{incolor}4}]:} \PY{n}{Image}\PY{p}{(}\PY{n}{filename}\PY{o}{=}\PY{l+s+s1}{\PYZsq{}}\PY{l+s+s1}{/Users/paha3326/main/Teaching/ASTR3720/hayne/lectures/lecture18/pressure\PYZhy{}gradient.png}\PY{l+s+s1}{\PYZsq{}}\PY{p}{,}\PY{n}{width}\PY{o}{=}\PY{l+m+mi}{250}\PY{p}{)}
\end{Verbatim}

\texttt{\color{outcolor}Out[{\color{outcolor}4}]:}
    
    \begin{center}
    \adjustimage{max size={0.9\linewidth}{0.9\paperheight}}{output_9_0.png}
    \end{center}
    { \hspace*{\fill} \\}
    

    \textbf{Figure 8.3:} Schematic representation of the pressure gradient
force. 'H' and 'L' refer to high and low pressure regions, respectively.
The pressure gradient in this case is negative, pointing in the opposite
direction of the \(x\)-coordinate.

\begin{center}\rule{0.5\linewidth}{\linethickness}\end{center}

    \subsubsection{8.1.3 Centrifugal
acceleration}\label{centrifugal-acceleration}

Centrifugal accelerations occur due to the tendency of a moving object
to keep to its "flight path" (i.e., Newton's First Law). In a rotating
frame, a force must be applied to keep the object "stationary", which is
really a curved path. For circular motion with angular velocity
\(\Omega\) and radius \(r\), the centrifugal acceleration is

\begin{align}
    \vec{a}_\mathrm{centrif.} &= \Omega^2 \vec{r}
\end{align}

The formula above shows that the acceleration is directed away from the
rotation axis. Examples of atmospheric phenomena where the centrifugal
force is important include small-scale vortices, such as tornados and
dust devils. In these cases, \(\vec{r}\) is the position vector from the
center of the vortex.

On larger scales, the centrifugal force due to planetary rotation may be
important. In the case of an air parcel at latitude \(\phi\) and
longitude \(\lambda\) on a planetary body with radius \(R\), the
distance to the rotation axis is \(r = R\cos \phi\). The magnitude of
the centrifugal acceleration is

\begin{align}
    a_\mathrm{centrif.} = \Omega^2 R\cos \phi
\end{align}

In vector form,

\begin{align}
    \vec{a}_\mathrm{centrif.} &= -\vec{\Omega} \times \left(\vec{\Omega} \times \vec{R}\right) \\
    &= \Omega^2R \begin{pmatrix} \cos\phi\cos\lambda \\ ~\cos\phi \sin\lambda \\ 0 \end{pmatrix}
\end{align}

where
\(\vec{R} = (x,y,z) = R(\cos\phi\cos\lambda, \cos\phi\sin\lambda, \sin\phi)\)
is the radius vector from the planet's center to the air parcel, and
\(\vec{\Omega} = (0, 0, \Omega)\) is the rotation vector. Again, we see
from the above vector equation that the centrifugal acceleration is
directed away from the planet's axis of rotation.

The magnitude of the centrifugal force due to planetary rotation is
\(\Omega^2 R\). This is typically small compared to \(g\). For example,
on Earth, we have
\(\Omega^2 R = (7.3\times 10^{-5}~\mathrm{s^{-1}})^2(6.4\times 10^6~\mathrm{m}) \approx 0.03\).
Compared to \(g = 9.8~\mathrm{m~s^{-2}}\), this is a correction of order
\(\Omega^2 R/g \sim 10^{-3}\).

\paragraph{Planetary flattening}\label{planetary-flattening}

When we measure the downward acceleration of "gravity" acting on a
stationary mass \(m\) on the surface of the Earth, we are really
measuring the sum of the gravitational acceleration and the centrifugal
acceleration due to Earth's rotation. Therefore, we define \(\vec{g}'\)
to include both forces:
\(\vec{g}' = \vec{g} + \vec{a}_\mathrm{centrifugal}\). Which direction
does \(\vec{g}'\) point? The answer is "down", of course. But we showed
above that there is a component of \(\vec{g}'\) due to the centrifugal
acceleration. Therefore, the "down" direction is the vector sum of the
planet's gravitational acceleration and the centrifugal acceleration due
to its rotation. At the equator, the \(g'\)-vector still points towards
the center of the planet (Why?). At other latitudes, the \(g'\)-vector
points slightly away from the direction towards the planet's center.
This leads to \emph{polar flattening}, where the equator bulges outward
relative to the poles (Fig. 8.4).

    \begin{Verbatim}[commandchars=\\\{\}]
{\color{incolor}In [{\color{incolor}5}]:} \PY{n}{Image}\PY{p}{(}\PY{n}{filename}\PY{o}{=}\PY{l+s+s1}{\PYZsq{}}\PY{l+s+s1}{/Users/paha3326/main/Teaching/ASTR3720/hayne/lectures/lecture18/polar\PYZhy{}flattening.png}\PY{l+s+s1}{\PYZsq{}}\PY{p}{,} \PY{n}{width}\PY{o}{=}\PY{l+m+mi}{250}\PY{p}{)}
\end{Verbatim}

\texttt{\color{outcolor}Out[{\color{outcolor}5}]:}
    
    \begin{center}
    \adjustimage{max size={0.9\linewidth}{0.9\paperheight}}{output_12_0.png}
    \end{center}
    { \hspace*{\fill} \\}
    

    \textbf{Figure 8.4:} Polar flattening in a fluid planet occurs due to
difference in the centrifugal acceleration
(\(\vec{a}_\mathrm{centrif.}\)) from equator to pole. Near the pole,
\(\vec{a}_\mathrm{centrif.}\) is directed tangential to the surface,
while at the equator, it is normal to the surface.
\(\vec{a}_\mathrm{centrif.}\) always points perpendicular to the axis of
rotation, and is strongest at the equator. (Vector magnitudes are not to
scale.)

\begin{center}\rule{0.5\linewidth}{\linethickness}\end{center}

    Flattening (also called: \emph{oblateness}, \emph{rotational
ellipticity}) is measured as the polar compression relative to a sphere:

\begin{align}
    f_\mathrm{r} = \frac{A-C}{A}
\end{align}

where \(A\) and \(C\) are the equatorial and polar radii of the rotating
planet, respectively. For a spherical planet, \(f_\mathrm{r} = 0\).
Higher rotation rates lead to greater flattening, if the planet can
deform to its potential surface. Gas giant planets are fluid, but
terrestrial planets have some rigidity, which can resist flattening.
Some values of \(f_\mathrm{r}\) for several planets are given in the
table below.

\begin{longtable}[]{@{}lccc@{}}
\toprule
Planet & Rotation Period (hr) & \(\Omega\) (rad s\(^{-1}\)) &
\(f_\mathrm{r}\)\tabularnewline
\midrule
\endhead
Mercury & 1408 & 1.2\(\times 10^{-6}\) & 0.0000\tabularnewline
Venus & -5833 & -3.0\(\times 10^{-6}\) & 0.0000\tabularnewline
Earth & 23.94 & 7.3\(\times 10^{-5}\) & 0.0034\tabularnewline
Mars & 24.62 & 7.1\(\times 10^{-5}\) & 0.0059\tabularnewline
Jupiter & 9.93 & 1.8\(\times 10^{-4}\) & 0.0649\tabularnewline
Saturn & 10.66 & 1.6\(\times 10^{-4}\) & 0.0980\tabularnewline
Uranus & -17.24 & -1.0\(\times 10^{-4}\) & 0.0229\tabularnewline
Neptune & 16.11 & 1.1\(\times 10^{-4}\) & 0.0171\tabularnewline
\bottomrule
\end{longtable}

    \subsubsection{8.1.4 Coriolis acceleration}\label{coriolis-acceleration}

Coriolis forces arise due to conservation of angular momentum,
\(\vec{L}\): displacements relative to the planet's rotation axis result
in accelerations in order to keep \(\vec{L}\) constant in the inertial
frame. The rotation vector \(\vec{\Omega}\) is aligned with the planet's
spin axis and has magnitude equal to its angular velocity,
\(\Omega = 2\pi/P\) (rad s\(^{-1}\)). Here, \(P\) is the planet's
rotation period.

As a simplified illustration of the Coriolis acceleration, consider the
case shown in Fig. 8.5. Here, a mass moves towards the axis of rotation
of a disk. We call the angular velocity of the mass \(\omega\) to
distinguish it from the disk's angular velocity, \(\Omega\). Initially,
\(\omega = \Omega\), but as the mass moves, its angular velocity changes
in order to conserve angular momentum. The angular momentum of the mass
\(m\) is \(L = mr^2\Omega\), and its rate of change is

\begin{align}
    \frac{dL}{dt} &= \frac{d}{dt}(mr^2\omega) \\
    &= -2mr\omega v + mr^2\frac{d\omega}{dt} \\
\end{align}

where we used the fact that the \(dr/dt = -v\). Setting \(dL/dt = 0\)
and \(\omega|_{t=0} = \Omega\), we have

\begin{align}
    2\Omega v &= r\frac{d\omega}{dt} \\
    &= a_x
\end{align}

That is, the tangential acceleration \(a_x = du/dt = rd\omega/dt\) is
equal to \(2\Omega v\). In other words, in the case where the velocity
of the displaced mass is toward the axis of rotation, the Coriolis
acceleration is tangential (and \emph{prograde}) to the circular motion.

    \begin{Verbatim}[commandchars=\\\{\}]
{\color{incolor}In [{\color{incolor}6}]:} \PY{n}{Image}\PY{p}{(}\PY{n}{filename}\PY{o}{=}\PY{l+s+s1}{\PYZsq{}}\PY{l+s+s1}{/Users/paha3326/main/Teaching/ASTR3720/hayne/lectures/lecture18/coriolis.png}\PY{l+s+s1}{\PYZsq{}}\PY{p}{,} \PY{n}{width}\PY{o}{=}\PY{l+m+mi}{250}\PY{p}{)}
\end{Verbatim}

\texttt{\color{outcolor}Out[{\color{outcolor}6}]:}
    
    \begin{center}
    \adjustimage{max size={0.9\linewidth}{0.9\paperheight}}{output_16_0.png}
    \end{center}
    { \hspace*{\fill} \\}
    

    \textbf{Figure 8.5:} A simplified illustration of the Coriolis effect. A
mass at the edge of a disk of radius \(r\) moves inward with initial
velocity \(v\) and angular velocity \(\Omega\). The Coriolis force in
this case is directed in the tangential direction, causing the mass to
follow a curved path. In a similar way, poleward air flows in planetary
atmospheres speed up in the zonal direction in order to conserve angular
momentum.

\begin{center}\rule{0.5\linewidth}{\linethickness}\end{center}

    A similar, but more involved derivation shows that the Coriolis
acceleration on a planetary body with rotation vector \(\vec{\Omega}\)
is

\begin{align}
    \vec{a}_\mathrm{coriolis} = -2\vec{\Omega}\times\vec{v}
\end{align}

where \(\vec{v} = (u,v,w)\) is the velocity vector. The resultant
accelerations are

\begin{align}
    a_x &= 2\Omega(\sin\phi~v - \cos\phi~w) \\
    a_y &= -2\Omega\sin\phi~u \\
    a_z &= 2\Omega\cos\phi~u
\end{align}

On length scales in planetary atmospheres where Coriolis forces are
significant, the vertical component of the velocity, \(w\) is negligible
compared to the horizontal components, \(u\) and \(v\). Also, the
vertical component of the Coriolis acceleration, \(a_z\) is typically
negligible compared to gravity. Defining the \emph{Coriolis parameter},
\(f \equiv 2\Omega\sin\phi\), we have the simplified equations

\begin{align}
    a_x &= fv \\
    a_y &= -fu \\
    a_z &\approx 0 \\
    \\
    \mathrm{or} \\
    \vec{a}_\mathrm{coriolis} &= f\begin{pmatrix} v \\ -u \\ 0 \end{pmatrix}
\end{align}

\subsubsection{8.1.5 Equations of motion in the rotating
frame}\label{equations-of-motion-in-the-rotating-frame}

From the force balance equations (Section 8.1.1) we can write the vector
equation of motion for a fluid parcel in a rotating reference frame:

\begin{align}
    \vec{a} = \frac{d}{dt}\vec{v} &= ~~(\mathrm{gravity}) + (\mathrm{pressure}) + (\mathrm{friction})~~~~~~~~+ ~~~~~~~(\mathrm{Coriolis}) + (\mathrm{centrifugal}) \\
    &= ~~~~~\vec{g} ~~~~~~~~~ - \frac{1}{\rho}\vec{\nabla}p ~~~~~~~~ + ~~\vec{a}_\mathrm{f}~~~~~~~~~~~~~~~~~- ~~~~~~~2\vec{\Omega}\times\vec{v} ~~~~- \vec{\Omega}\times(\vec{\Omega}\times\vec{R}) \\
\end{align}

For large-scale atmospheric motions, we can define the apparent gravity
vector including the centrifugal acceleration, as described above. Since
\(g'\) points opposite to the local vertical direction \(\hat{k}\), we
have
\(\vec{g}' = \vec{g} - \vec{\Omega}\times(\vec{\Omega}\times\vec{R}) = -g'\hat{k}\).
In this case,

\begin{align}
    \frac{d\vec{v}}{dt} &= -g'\hat{k} - \frac{1}{\rho}\vec{\nabla}p - 2\vec{\Omega}\times\vec{v} + \vec{a}_\mathrm{f}
\end{align}

    \subsection{8.2 Balanced flow}\label{balanced-flow}

Often we can neglect terms in the equations of motion, when their
magnitudes are small compared to the dominant terms. For example, winds
in the mid-latitudes of rapidly rotating planets, such as Earth, are
dominated by Coriolis- and pressure-gradient accelerations. Slowly
rotating planets, like Venus, tend to be dominated by centrifugal- and
pressure-gradient forces. Friction tends to be important at smaller
scales and near to the surface. Below, we discuss several important
approximations for balanced flow.

\subsubsection{8.2.1 Rossby number}\label{rossby-number}

To estimate the importance of planetary rotation on the dynamics of a
flow, we use the \emph{Rossby number}, which is the ratio of the net
acceleration to the Coriolis acceleration:

\begin{align}
\operatorname{Ro} = \frac{\mathrm{Magnitude~of~net~acceleration}}{\mathrm{Magnitude~of~Coriolis~acceleration}}
\end{align}

A small Rossby number (i.e., \(\operatorname{Ro} \ll 1\)) indicates that
Coriolis forces dominate the other terms in the force balance. We define
characteristic dimensions of the flow feature (e.g., a tropical cyclone,
tornado, or the jet stream), having horizontal length scale \(L\) and
velocity \(U\). The time for the flow to cross the characteristic length
scale is \(t \sim L/U\). The magnitude of the net acceleration is then
\(|d\vec{v}/dt| \sim U/t = U^2/L\), and we can show that
\(|\vec{a}_\mathrm{cor.}| \sim fU\). Therefore, the Rossby number is

\begin{align}
    \operatorname{Ro} = \frac{U^2/L}{fU} = \frac{U}{fL}
\end{align}

We see that the Rossby number increases towards the equator (since \(f\)
decreases equator-ward), indicating that planetary rotation more
strongly affects dynamics at higher latitudes. More explicitly,

\begin{align}
    \lim_{\phi \rightarrow 0^\circ} \operatorname{Ro} &= \infty ~~~~~\mathrm{(Coriolis~negligible)}\\
    \lim_{\phi \rightarrow 90^\circ} \operatorname{Ro} &= \frac{U}{2\Omega L} ~~~~~(\mathrm{Coriolis~dominant~if} U<2\Omega L)
\end{align}

    \begin{Verbatim}[commandchars=\\\{\}]
{\color{incolor}In [{\color{incolor}8}]:} \PY{n}{Image}\PY{p}{(}\PY{n}{filename}\PY{o}{=}\PY{l+s+s1}{\PYZsq{}}\PY{l+s+s1}{/Users/paha3326/main/Teaching/ASTR3720/hayne/lectures/lecture18/pressure\PYZhy{}gradient\PYZus{}geostrophic.png}\PY{l+s+s1}{\PYZsq{}}\PY{p}{,} \PY{n}{width}\PY{o}{=}\PY{l+m+mi}{400}\PY{p}{)}
\end{Verbatim}

\texttt{\color{outcolor}Out[{\color{outcolor}8}]:}
    
    \begin{center}
    \adjustimage{max size={0.9\linewidth}{0.9\paperheight}}{output_20_0.png}
    \end{center}
    { \hspace*{\fill} \\}
    

    \textbf{Figure 8.6:} Geostrophic flow around centers of low and high
pressure in a planet's northern hemisphere. (A) Flow is initially down
the pressure gradient, i.e., inward for the low-pressure system, and
outward for the high-pressure system. (B) Coriolis forces divert winds
to the right in the northern hemisphere. (C) Balance of the pressure
gradient and Coriolis accelerations results in counter-clockwise
(cyclonic) flow around the low-pressure system, and clockwise
(anticyclonic) flow around the high-pressure system. Geostrophic winds
between the two systems point with the lower pressure on the left.

\begin{center}\rule{0.5\linewidth}{\linethickness}\end{center}

    \subsubsection{8.2.2 Geostrophic balance}\label{geostrophic-balance}

If a flow feature's horizontal scale \(L_\mathrm{h}\) is much larger
than its vertical length scale \(L_\mathrm{v}\), then vertical
accelerations can be neglected. This is the case for large-scale
atmospheric motions on Earth, where
\(H \sim L_\mathrm{v} \ll L_\mathrm{h} \sim R\), where \(R\) is the
planetary radius and \(H\) is the scale height. Neglecting friction, the
equations of motion simplify:

\begin{align}
    \frac{d\vec{v}}{dt} = \begin{pmatrix} \frac{du}{dt} \\ \frac{dv}{dt} \\ 0 \end{pmatrix} &= \begin{pmatrix} fv - \frac{1}{\rho}\frac{\partial p}{\partial x} \\ -fu - \frac{1}{\rho}\frac{\partial p}{\partial y} \\ -g - \frac{1}{\rho}\frac{\partial p}{\partial z} \end{pmatrix}
\end{align}

where again, \(f = 2\Omega\sin\phi\) is the Coriolis parameter. Note
that the equation of vertical motion is reduced to hydrostatic balance:
\(\partial p/\partial z = -\rho g\).

Over the characteristic timescale \(L/U\), the net acceleration on a
balanced flow is zero: \(du/dt = dv/dt = 0\). Provided the conditions
\(\operatorname{Ro} \ll 1\) and \(L_\mathrm{v} \ll L_\mathrm{h}\) are
met, the two horizontal equations of motion reduce to the geostrophic
wind equations:

\begin{align}
    u_\mathrm{g} = -\frac{1}{f\rho}\frac{\partial p}{\partial y}, ~~~ v_\mathrm{g} = \frac{1}{f\rho}\frac{\partial p}{\partial x}
\end{align}

where the subscripts "\(\mathrm{g}\)" represent geostrophic flow. From
these two simple equations, we can infer four properties of the
geostrophic wind:

\begin{enumerate}
\def\labelenumi{\arabic{enumi}.}
\tightlist
\item
  Flow is directed tangential to the pressure gradient.
\item
  The lower pressure is on the left of the wind vector in the northern
  hemisphere.
\item
  Wind speed is proportional to the pressure gradient.
\item
  Wind speed increases toward the equator as \(f \rightarrow 0\).
\end{enumerate}

Figure 8.6 illustrates how flows develop around low- and high-pressure
systems through geostrophic balance. Flows around low- and high-pressure
systems are \emph{cyclonic} and \emph{anticyclonic}, respectively.
Cyclonic flow is counter-clockwise in the northern hemisphere, and
clockwise in the southern hemisphere. By measuring the wind velocity
vectors of weather systems in geostrophic balance, we can infer pressure
gradients (Fig. 8.7).

    \begin{Verbatim}[commandchars=\\\{\}]
{\color{incolor}In [{\color{incolor}10}]:} \PY{n}{Image}\PY{p}{(}\PY{n}{filename}\PY{o}{=}\PY{l+s+s1}{\PYZsq{}}\PY{l+s+s1}{/Users/paha3326/main/Teaching/ASTR3720/hayne/lectures/lecture18/jupiter\PYZhy{}grs\PYZus{}neptune\PYZhy{}gds\PYZus{}winds.png}\PY{l+s+s1}{\PYZsq{}}\PY{p}{,}\PY{n}{width}\PY{o}{=}\PY{l+m+mi}{500}\PY{p}{)}
\end{Verbatim}

\texttt{\color{outcolor}Out[{\color{outcolor}10}]:}
    
    \begin{center}
    \adjustimage{max size={0.9\linewidth}{0.9\paperheight}}{output_23_0.png}
    \end{center}
    { \hspace*{\fill} \\}
    

    \textbf{Figure 8.7} Vortices on giant planets: the "little red spot" on
Jupiter (\emph{Cheng et al., 2008}) and Neptune's "great dark spot"
(schematic). Both features are located at \(\sim 20^\circ\)S latitude on
their respective planets, and have \(\operatorname{Ro} \ll 1\). Are
these cyclonic or anticyclonic features? Do they have low- or
high-pressure centers?

\begin{center}\rule{0.5\linewidth}{\linethickness}\end{center}

    \subsubsection{8.2.3 Cyclostrophic balance}\label{cyclostrophic-balance}

A localized spinning air mass, such as a tornado, may by unaffected by
planetary rotation (\(\operatorname{Ro} > 1\)) due to its small size.
\emph{Cyclostrophic balance} occurs when a strong local pressure
gradient is balanced against the centrifugal force. For a tornado with
radius \(r\) and tangential wind speed \(v_\mathrm{t} = r\omega\),
cyclostrophic balance implies

\begin{align}
    \frac{v_t^2}{r} = \omega^2 r = \frac{1}{\rho}\frac{\partial p}{\partial r}
\end{align}

For a given pressure drop, \(\delta p\) and scale, \(\delta r\) of the
tornado, we can estimate the tangential velocity:

\begin{align}
    v_t \approx \sqrt{ \frac{\delta r}{\rho}\frac{\delta p}{\delta r} } \sim \sqrt{\frac{\delta p}{\rho}}
\end{align}

On a planetary scale, cyclostrophic balance occurs when the planetary
Rossby number is large, and yet the atmosphere itelf rotates rapidly.
For example, Venus' slow planetary rotation gives
\(\operatorname{Ro} \sim 10\) for the mid-latitudes, yet its
super-rotating atmosphere experiences signficant centrifugal forces.


    % Add a bibliography block to the postdoc
    
    
    
    \end{document}
